\documentclass[10pt,a4paper,twocolumn]{scrartcl}
\usepackage[english]{babel}
\usepackage[utf8]{inputenc}
\usepackage[official]{eurosym}
\usepackage[T1]{fontenc}

%Tables with cells spanning multiple rows
\usepackage{multirow}

% Til at indsætte flot formateret kode:
\usepackage{listings}
%\usepackage{color}
\usepackage[usenames,dvipsnames]{xcolor}

\usepackage{lmodern}

%Marginer:
\usepackage[a4paper,
			left=1.5cm,
			right=1.5cm,
			top=3cm,
			bottom=3cm]{geometry}

% Seperation between columns (default: 10pt)
%\setlength{\columnsep}{15pt}

%\usepackage{icomma}

\usepackage{graphicx,epic,eepic}
%\usepackage[pdftex]{graphicx}
\usepackage{amssymb,amsmath}%,amsthm}
\usepackage{latexsym}
\usepackage{mathrsfs}
\usepackage{verbatim} %man kan bruge \begin{comment}...\end{comment}
\usepackage{mathtools} %mathclap kan benyttes til stackrel
\usepackage{xspace}						% Add space after text macros with \xspace
\usepackage{bm}

%Kemiske formler: \ce{<formel>}
\usepackage[version=3]{mhchem}

\usepackage{siunitx}
%\sisetup{separate-uncertainty}
\sisetup{
    separate-uncertainty=true,
	list-separator = {\text{, }},
    list-final-separator = {\text{, and }},
    round-mode = figures,
    round-precision = 3
}


%Flotte tabeller:
\usepackage{array,booktabs}

%\numberwithin{align}{section}

%Caption laves anderledes:
%\usepackage[font=small,labelfont={sf,bf},labelsep=space,width=.9\textwidth,indention=.5cm]{caption}
\usepackage[format=plain,font=small,labelfont={sf,bf},labelsep=space,width=.9\textwidth]{caption}
%\usepackage{caption}
%\captionsetup{font=small, labelfont=bf}
%\setlength{\captionmargin}{20pt}

%flere figurer ved siden af hinanden:
\usepackage{subfig}

\usepackage{float}
%\usepackage{fullpage}

%overfull hbox under 3pt ignoreres:
%\hfuzz=3pt

%\newcommand{\e}{\mathrm{e}}
\newcommand{\R}{\mathbb{R}}
\newcommand{\N}{\mathbb{N}}
\newcommand{\Z}{\mathbb{Z}}
\newcommand{\Q}{\mathbb{Q}}
\newcommand{\vf}{\varphi}
\newcommand{\ve}{\varepsilon}
\newcommand{\ul}{\underline}
\newcommand{\dul}[1]{\underline{\underline{#1}}}
\newcommand{\ph}{\phantom{}}
\newcommand{\Ra}{\Rightarrow}
\newcommand{\La}{\Leftarrow}
\newcommand{\Lra}{\Leftrightarrow}
\newcommand{\ra}{\rightarrow}
\newcommand{\la}{\leftarrow}
\newcommand{\nn}{\nonumber}
\newcommand{\ol}{\overline}
\newcommand{\mb}{\mathbf}
\newcommand{\bs}{\boldsymbol}
\newcommand{\mc}{\mathcal}
\newcommand{\p}{\partial}
\newcommand{\I}{\mathbf{i}}
\newcommand{\J}{\mathbf{j}}
\newcommand{\K}{\mathbf{k}}
\newcommand{\pd}[2]{\frac{\partial #1}{\partial #2}}
\newcommand{\pdn}[3]{\frac{\partial^{#1} #2}{\partial {#3}^{#1}}}
\newcommand{\pdc}[3]{\left(\frac{\partial #1}{\partial #2}\right)_{#3}}
\newcommand{\pds}[3]{\frac{\partial^2 #1}{\partial #2 \partial #3}}
\newcommand{\avg}[1]{\langle #1 \rangle}
\newcommand{\GZ}{\mathscr{Z}}
\newcommand{\fpe}{4\pi\varepsilon_0}
\newcommand{\ifpe}{\frac{1}{4\pi\varepsilon_0}}
\newcommand{\E}[1]{\cdot 10^{#1}}
\newcommand{\phm}{\phantom{maple}}
\newcommand{\un}[1]{\mathbf{\hat{#1}}}
\newcommand{\us}[1]{\boldsymbol{\hat{#1}}}
%\newcommand{\ns}{\setlength{\itemsep}{0cm}\setlength{\parskip}{0cm}}
\newcommand{\ti}[1]{\tilde{#1}}
\newcommand{\co}{\textup{count}}

\newcommand{\ket}[1]{| #1 \rangle}
\newcommand{\bra}[1]{\langle #1 |}
\newcommand{\ip}[2]{\langle #1 | #2 \rangle}
\newcommand{\expv}[3]{\langle #1 | #2 | #3 \rangle}


%\renewcommand{\phi}{\varphi}
%\renewcommand{\epsilon}{\varepsilon}

\DeclareMathOperator{\id}{d\!}
\DeclareMathOperator{\dx}{d}
\DeclareMathOperator{\im}{Im}
\DeclareMathOperator{\re}{Re}
\DeclareMathOperator{\e}{e}
\DeclareMathOperator{\dist}{dist}
\DeclareMathOperator{\prob}{Prob}

\let\oldvec\vec
\renewcommand{\vec}[1]{\bm{\mathrm{#1}}}
\newcommand{\psvec}[1]{\tilde{\vec{#1}}}
\newcommand{\uvec}[1]{\hat{\vec{#1}}}
\newcommand{\gvec}[1]{\oldvec{#1}}
\newcommand{\mat}[1]{\bm{\mathrm{#1}}}
\newcommand{\transpose}{^\mathsf{T}}
\let\T\transpose

\newcommand{\mean}[1]{\bar{#1}}
\newcommand{\var}[1]{\sigma_{#1}^2}
\newcommand{\width}[1]{\sigma_{#1}}
\newcommand{\given}{\, | \,}
\newcommand{\constant}{\text{const.}}
\DeclarePairedDelimiter{\norm}{\lVert}{\rVert}

% Text formatting
\newcommand{\newterm}[1]{\textbf{#1}}
\newcommand{\library}[1]{\textsf{#1}}
\newcommand{\function}[1]{\texttt{#1}}
\newcommand{\class}[1]{\textsf{#1}}
\newcommand{\filename}[1]{\texttt{#1}}
\newcommand{\variable}[1]{\textit{\texttt{#1}}}
\newcommand{\code}[1]{\texttt{#1}}
\newcommand{\eg}{e.g.\xspace}
\newcommand{\ie}{i.e.\xspace}
\newcommand{\shark}{\library{Shark}\xspace}
\newcommand{\hl}[1]{{\color{Green!50!YellowGreen}#1}}

%Horizontale tykke streger via \HRule (Bruges til forsiden):
\newcommand{\HRule}{\rule{\textwidth}{1mm}}

%Skillelinjer:
\newcommand{\skillelinje}{\begin{center}\rule{0.9\textwitdh}{0.3mm}\end{center}}

%Orddelinger. Brug "- i stedet for - i teksten, hvis der skal være en bindestreg i et ord.
%\hyphenation{Maple-kom-man-do-en fer-mi-for-del-ings-funk-ti-on-en}

%linjeafstand:
\linespread{1.033}

%Fjerner linjeafstanden i alle lister
\usepackage{enumitem}
\setlist{noitemsep}

%FiXme-noter i margen:
\usepackage[draft,english,marginclue,footnote]{fixme}

%Laveste niveau i indholdsfortegnelse:
\setcounter{tocdepth}{3}
%Numereringsniveau:
\setcounter{secnumdepth}{2}

%Ingen indrykning ved nyt afsnit:
%\parindent=0pt

%Der vises labels i margin - bør kun anvendes i draft
%\usepackage[notref,notcite]{showkeys}

%Det sikres, at TOC (Table Of Contents) ikke selv er med i TOC, samt at litteraturlisten er med:
\usepackage[nottoc]{tocbibind}

%Klikbar indholdsfortegnelse m.m.
\usepackage{hyperref}
\hypersetup{colorlinks,
  citecolor=black,
  linkcolor=black,
  urlcolor=black
}

%Include time of compilation:
\usepackage{time}

\usepackage[round,semicolon]{natbib}	% Natural science refereneces

%Bedre citationer med BibTeX:
\usepackage{cite}

%Environment til at lave citater med engelsk orddeling
%\newenvironment{engquote}%
%	{\begin{quote}\selectlanguage{english}\small}%
%	{\end{quote}\selectlanguage{danish}}

%New quote environment
\renewenvironment{quote}%
	{\begin{quote}\small}%
	{\end{quote}}

\newcommand{\question}[1][]{
	\ifthenelse{\isempty{#1}}
	{\section{}}
	{\section{#1}}
}

%fancyhdr - sidehoved og sidefod
\usepackage{lastpage}
%\usepackage{fancyhdr}
%%\pagestyle{empty}
%\pagestyle{fancy}
\usepackage{fancyhdr}
\setlength{\headheight}{14.5pt}
\pagestyle{fancy}

%Indholdet af sidehoved og sidefod:
%\renewcommand{\headheight}{14.5pt} %er obligatorisk v. fancyhdr
\renewcommand{\headrulewidth}{0.5pt}
\renewcommand{\footrulewidth}{0.5pt}
%\renewcommand{\sectionmark}[1]{\markright{\thesection\ #1}}
%\lhead{\rightmark} 
%\lhead{\textsc{Optisk spektrum af natrium-atomet}} 
%\rhead{\textsc{Den kopernikanske revolution}}
%\lfoot{bla}
%\cfoot{Side \thepage\ af \pageref{LastPage}}
%\rfoot{bla}

\newcommand{\doctitle}{Notes on Scientific Writing}
%\newcommand{\assignment}{Exam Assignment}

\fancyhf{}
\fancyhead[L]{\textit{\doctitle}} 
%\fancyhead[R]{\textsc{\coursename}}
\fancyfoot[C]{Page \thepage\ of \pageref{LastPage}}

\title{\doctitle}
\author{}
\date{}

\begin{document}
\maketitle

\section{Criteria for Effective Writing}
\begin{description}
    \item[Clear] Reader gets the message
    \item[Complete] Reader's questions are answered
    \item[Correct] Message is accurate
    \item[Concise] Saves reader time
\end{description}

\section{Coherence}
\emph{Coherence} refers to the logical sequence of sentences within a paragraph. 

Make sure it is clear\dots
\begin{itemize}
    \item \dots where did it come from?
    \item \dots where is it going?
\end{itemize}

\subsection{How to Improve Coherence}
\begin{itemize}
    \item Put similar topics together.
        \begin{itemize}
            \item Go through the text, identify the theme of each paragraph and colour code them in the margin with a marker.
            \item Collect similar paragraph.
            \item Rewrite/delete repeating text.
        \end{itemize}
        
    \item Discuss \emph{one} idea within a paragraph.
    \item State the idea in a ``topic sentence''.
    \item Put topic sentence \emph{first} in the paragraph.
        \begin{itemize}
            \item \emph{Example:} ``Bees spread pollen\dots'' (paragraph about bees); ``Pollen is spread by bees\dots'' (paragraph about pollen).
        \end{itemize}
        
    \item Use \emph{new-old construction} -- repetition can be good!
        \begin{itemize}
            \item Move $A \ra C$ by going $A \ra B$ and then $B \ra C$.
            \item Start the sentence $\ra$ introduce the \emph{new} thing. Period. Start with the now \emph{old} thing $\ra$ go to the end.
            \item That is, end with the \emph{new}, start with the \emph{old}.
            \item \emph{Example:} ``Before analysis, the images are \emph{segmented}. The \emph{segmentation} involves\dots''
        \end{itemize}
        
    \item Be \emph{consistent} in your style.
        \begin{itemize}
            \item Use the same \emph{word} for the same idea. E.g.\ decide on either \emph{experiment, trial, study} etc.
            \item Use same \emph{organisational pattern} for successive sentences and paragraphs.
        \end{itemize}
        
    \item Use \emph{parallel structure}.
        \begin{itemize}
            \item Balance nouns with nouns, (ad)verbs wit (ad)verbs, prepositions with prepositions.
            \item Use copy and past -- it is \emph{not} easier for the reader if you vary the construction!
                \begin{description}
                    \item[Example] For subset 1, RMSE {\color{blue!80}was} {\color{Green!80}less} {\color{BrickRed!80}for} class A {\color{Purple!80}when} $\theta < 1$ {\color{Green!80}than} {\color{Purple!80}when} $\theta > 1$, {\color{Orange!80}whereas} training time {\color{blue!80}was} {\color{Green!80}faster} {\color{BrickRed!80}for} class A {\color{Purple!80}when} $\theta < 1$ {\color{Green!80}than} {\color{Purple!80}when} $\theta > 1$.

                    For subset 2, {\color{Orange!80}however}, RMSE {\color{blue!80}was} {\color{Green!80}greater} {\color{BrickRed!80}for} class A {\color{Purple!80}when} $\theta < 1$ {\color{Green!80}than} {\color{Purple!80}when} $\theta > 1$, {\color{Orange!80}whereas} training time {\color{blue!80}was} {\color{Green!80}slower} {\color{BrickRed!80}for} class A {\color{Purple!80}when} $\theta < 1$ {\color{Green!80}than} {\color{Purple!80}when} $\theta > 1$.
                \end{description}
                
        \end{itemize}
        
\end{itemize}

\section{Structure}
\subsection{Title}
The title should tell what the paper is about!
\begin{description}
    \item[Informative] Describe the research. Avoid jargon.
    \item[Specific] Differentiate the research from other research on the subject. Use specific words, not broad ones.
    \item[Concise] Say only what is necessary -- use 7 to 10 words. Use most important words related to topic.
\end{description}

\subsubsection*{Guidelines}
\begin{itemize}
    \item The title should be \emph{descriptive}, not \emph{declarative}:
        \begin{description}
            \item[Descriptive] Tells the reader the \emph{objective} of the research.
            \item[Declarative] Tells the reader the \emph{results} or \emph{conclusion} of the research.
        \end{description}

    \item Avoid waste words:
        \begin{itemize}
            \item Observations of\dots
            \item Studies of\dots
            \item Investigations of\dots
            \item Examination of\dots
            \item Research on\dots
            \item Note about\dots
            \item Preliminary study of\dots
        \end{itemize}
    \item Avoid abbreviations and acronyms.
    \item Avoid catchy or sexy titles.
    \item Minimize punctuation (commas and colons are OK).
\end{itemize}

\subsection{Abstract}
\begin{itemize}
    \item Make the abstract stand alone.
        \begin{itemize}
            \item No bibliographic references.
            \item No references to tables, figures, or text.
            \item Limited use of abbreviations. If it is not a ``standard'' abbreviation, it should be defined at first use.
        \end{itemize}
        
    \item Make it informative, not indicative:
        \begin{description}
            \item[Informative] States objectives or aim, summarises methods and results, and supports conclusion with data. Makes the reader want to read your paper because it \emph{is} interesting.
            \item[Indicative] Indicates objectives of the research and suggest results in general terms (``Results suggest that the method improved the performance.'') Makes the reader want to read your paper because it \emph{might} interesting.
        \end{description}
\end{itemize}

\subsubsection*{Guidelines}
\begin{itemize}
    \item Structure:
        \begin{itemize}
            \item Start with motivation, if space allows.
            \item Clearly state objective, aim, or purpose.
            \item Describe essential methods.
            \item Summarize important results.
            \item End with important conclusions and impact.
        \end{itemize}

    \item Be clear, specific and concise!
    \item Use \emph{copy and paste}!
        \begin{itemize}
            \item Ensures consistent style and organization.
            \item No contradiction between text and abstract.
        \end{itemize}
        
\end{itemize}


\section{General Recommendations}
\subsection{Verb Tenses for Papers}
\begin{description}
    \item[Motivation] Present.
    \item[Literature] Past or present perfect (``Studies have shown that\dots'').
    \item[Objective] Past.
    \item[Methods] Past.
    \item[Results and Discussion] Past/present (``Training time \hl{was} faster \dots, which \hl{suggests}\dots'').
    \item[Conclusions] Present.
\end{description}



\end{document}
